\documentclass{article}

\usepackage{fancyhdr} % Required for custom headers
\usepackage{lastpage} % Required to determine the last page for the footer
\usepackage{extramarks} % Required for headers and footers
\usepackage{graphicx} % Required to insert images
\usepackage{tikz}
\usepackage{alltt}
\usepackage{url}

\usetikzlibrary{shapes.multipart,positioning}

% Margins
\topmargin=-0.45in
\evensidemargin=0in
\oddsidemargin=0in
\textwidth=6.5in
\textheight=9.0in
\headsep=0.25in 

\linespread{1.1} % Line spacing

% Set up the header and footer
\pagestyle{fancy}
\chead{\hmwkClass : \hmwkTitle} % Top center header
\rhead{\firstxmark} % Top right header
\lfoot{\lastxmark} % Bottom left footer
\cfoot{} % Bottom center footer
\rfoot{Page\ \thepage\ of\ \pageref{LastPage}} % Bottom right footer
\renewcommand\headrulewidth{0.4pt} % Size of the header rule
\renewcommand\footrulewidth{0.4pt} % Size of the footer rule

\newcommand{\hmwkTitle}{Homework\ \#3} % Assignment title
\newcommand{\hmwkClass}{Advanced Networking,\ Fall 2016} % Course/class

\tikzset{
    font=\sffamily,
    BLOCK/.style={
        draw,
        align=center,
        text height=0.4cm,
        draw=gray!50,
        fill=gray!20,
        rectangle split, 
        rectangle split horizontal,
        rectangle split parts=#1, 
    }
}

\title{foo}

%-------------------------------------------------------------------------------

\begin{document}

%\maketitle


%-------------------------------------------------------------------------------
%   PROBLEM 1
%-------------------------------------------------------------------------------

% To have just one problem per page, simply put a \clearpage after each problem

\section*{Assignment}


We already have some hand-on experience with P4 in the class by implementing
source routing protocol. In this assignment, we are going to do the opposite
way. We will implement a \emph{Traceroute} protocol that each packets is tagged
with the output port that it departed from a particular switch. For instance, a
packet from host h1 to h3 traverses through the switch s1 out of port 3 and
switch s3 also out of port 1. Then, the trace in the packet payload should show
that Packets have passed 2 hops and the list of ports which are 3 and 1.


To do this, you will write a P4 program and a \emph{commands.txt} file to do the
following:

\begin{enumerate}

\item Define the emph{traceroute_port} header that has only a emph{port} field.

\item Create instances of the traceroute\_port header. Note that, it can have a
\emph{num\_valid} of output ports. HINT: use header stack.

\item Define parsers for traceroute_head and traceroute_port.

\item Define \emph{add\_port} action which increases the number of valid ports,
adds a traceroute\_port header and sets its port field to the output port that
the packet will depart from.

\item Define the add_traceroute_head which adds the traceroute_head

\item Define a table\label{tbl} that will match on the Ethernet protocol and
will have two possible actions: \emph{add\_port} and
\emph{add\_traceroute\_head}.

\item Add the table that is define in~ref{tbl} to the processing pipeline.

\item Finally, you need to configure the table: if the packet matches the
TRACEROUTE\_PROTOCOL (0x6900), then we only need to execute the add\_port
action. Otherwise, we have execute the add\_traceroute\_head action.

You may have to look at the P4 specification for more details:
 (\url{http://p4.org/wp-content/uploads/2016/11/p4-spec-latest.pdf}):

\end{enumerate}

\section*{What to hand in}
\begin{itemize}
\item The P4 program (p4src/traceroute.p4).
\item The configuration file (commands.txt).
\end{itemize}


\end{document}
