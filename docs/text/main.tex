\documentclass{article}

\usepackage{fancyhdr} % Required for custom headers
\usepackage{lastpage} % Required to determine the last page for the footer
\usepackage{extramarks} % Required for headers and footers
\usepackage{graphicx} % Required to insert images
\usepackage{tikz}
\usepackage{alltt}
\usepackage{url}

\usetikzlibrary{shapes.multipart,positioning}

% Margins
\topmargin=-0.45in
\evensidemargin=0in
\oddsidemargin=0in
\textwidth=6.5in
\textheight=9.0in
\headsep=0.25in 

\linespread{1.1} % Line spacing

% Set up the header and footer
\pagestyle{fancy}
\chead{\hmwkClass : \hmwkTitle} % Top center header
\rhead{\firstxmark} % Top right header
\lfoot{\lastxmark} % Bottom left footer
\cfoot{} % Bottom center footer
\rfoot{Page\ \thepage\ of\ \pageref{LastPage}} % Bottom right footer
\renewcommand\headrulewidth{0.4pt} % Size of the header rule
\renewcommand\footrulewidth{0.4pt} % Size of the footer rule

\newcommand{\hmwkTitle}{Homework\ \#3} % Assignment title
\newcommand{\hmwkClass}{Advanced Networking,\ Fall 2016} % Course/class

\tikzset{
    font=\sffamily,
    BLOCK/.style={
        draw,
        align=center,
        text height=0.4cm,
        draw=gray!50,
        fill=gray!20,
        rectangle split, 
        rectangle split horizontal,
        rectangle split parts=#1, 
    }
}

\title{foo}

%-------------------------------------------------------------------------------

\begin{document}

%\maketitle


%-------------------------------------------------------------------------------
%   PROBLEM 1
%-------------------------------------------------------------------------------

% To have just one problem per page, simply put a \clearpage after each problem

\section*{Assignment}


In this assignment, you will use P4 to implement a custom protocol
that tracks the path UDP packets travel through a network. Your custom
traceroute header will look like this:

\begin{alltt}
 num\_valid (4 bytes) | port\_1 (1 byte) | port\_2 (1 byte) | ... | port\_n (1 byte) | payload
\end{alltt}

\noindent
When a switch receives an EasyTrace packet, it should append the outgoing port and increment num\_valid by 1. When the recipient receives the packet, it should print the sequence of ports that the packet traversed.

The TA has implemented much of the needed infrastructure for you, and all of the code is available:

\begin{alltt}
https://github.com/usi-systems/easytrace
\end{alltt}

\noindent
You will need to modify two files:

 \begin{itemize}
 \item \texttt{p4src/p4src/easytrace.p4}: contains the skeleton P4 program for EasyTrace.
 \item \texttt{commands.txt}: configures the EasyTrace tables.
 \end{itemize}

 To test your program, you can use the \texttt{run\_demo.sh} script to start Mininet with
 a test topology. On any of the hosts, you can use  \texttt{client\_server.py} script to send and receive packets.
 To run in server mode, execute the following command on any host:

 
\begin{alltt}
sudo ./client\_server.py -s
\end{alltt}

\noindent
 To run in client mode, execute the following command on any host:

\begin{alltt}
sudo ./client\_server.py -c <host>
\end{alltt}


\noindent
Note that packets will be forwarded by IP destination. To simplify, you will only need to append the
EasyTrace header to UDP packets. So, you will need to modify the P4 source
to parse IP, UDP, and EasyTrace headers.

We will overload the use of the IP protocol field
to indicate whether or not a packet contains an  EasyTrace header.
We will use a value of (0xFD) to indicate the presence of the header.

At the first hop switch, you should check to see if the packet
contains a EasyTrace header. If it does not (i.e., there is a value in
the protocol field other than 0xFD) then you should check if the
packet is a UDP packet. If it is (i.e., IP protocol is 0x11) then you can add an EasyTrace header, and
modify the IP protocol field to use the 0xFD value. The EasyTrace header should follow the UDP header, and
come before the packet payload.

At subsequent switches, you should check if the  packet contains a EasyTrace header. If it does,
increment the num\_valid field, and append the port number.

At the client program, you should print the list of ports that the packet passed through.

You may want to look at the P4 specification for reference:
  (\url{http://p4.org/wp-content/uploads/2016/11/p4-spec-latest.pdf}):



 \section*{What to hand in}
 \begin{itemize}
 \item The P4 program (p4src/easytrace.p4).
 \item The configuration file (commands.txt).
 \end{itemize}


\end{document}
